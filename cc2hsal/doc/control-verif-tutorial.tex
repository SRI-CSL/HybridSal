\documentclass{llncs}
%\usepackage{sriram-macros}
\usepackage{url}
\usepackage{amsmath,amssymb,amsfonts}
%\usepackage{electComp}
\pagestyle{plain}
\usepackage{graphicx}

\title{Cyber-Component Verification Using HybridSAL}
%  \thanks{Tiwari's work supported in part by DARPA under Contract No. FA8650-10-C-7078,
%  NSF grants CSR-0917398 and SHF:CSR-1017483.}
\author{Ashish Tiwari}
\institute{SRI International, Menlo Park, CA.\ \url{ashish.tiwari@sri.com}}

\newif\iftrversion\trversionfalse
\def\linODEabs{\tt{linODEabs}}
\newcommand\ignore[1]{{}}
\newcommand\ite{{\mathtt{ite}}}
\newcommand\True{{\mathtt{True}}}
\newcommand\False{{\mathtt{False}}}

\begin{document}

\maketitle

\begin{abstract}
  This tutorial describes the process of
  designing cyber-components of a complex
  system guided by formal verification.
\end{abstract}

\def\XX{\mathbb{X}}
\def\QQ{\mathbb{Q}}
\def\YY{\mathbb{Y}}
\def\RR{\mathbb{R}}
\def\ra{\rightarrow}
%\def\bra{{\crm{\ra}}}  % make it \ra_a
\def\bra{{\stackrel{a}{\ra}}}  % make it \ra_a
%\def\rra{{\cem{\ra}}}  % make it \ra_c
\def\rra{{\stackrel{c}{\ra}}}  % make it \ra_c

\section{The Tools}

The tools used in the verification of cyber-controllers
are the following:

\begin{description}
  \item[Matlab Simulink Stateflow:]
    We assume that the cyber-components are 
    designed using Matlab's Simulink Stateflow language.
    Any changes to the controller models are 
    performed using Matlab.
    \\
    Preferred version: R13b
  \item[Matlab extension for specifying LTL properties:]
    VU has developed scripts that 
    extend the graphical user interface
    of Matlab Simulink/Stateflow with dialog boxes 
    for specifying temporal logic properties using a pattern system.
    \\
    Desired properties of the controllers are
    attached to the Simulink subsystem or Stateflow chart
    using this extension.
  \item[MDL2MGACyber.exe:]
    VU has developed a tool that translates a Matlab model
    (exported as a .mdl file) into an XML representation
    (cyber-composition language).
    Both the system and the LTL property are 
    included in the XML file.
    \\
    This tool is part of META release.
  \item[cybercomposition2hsal:]
    SRI has developed a tool that translates models in
    the cyber composition language (.xml files) into
    the HybridSal formal verification language.
    \\
    This tool is currently available as a separate
    script. It will soon be integrated into the META release.
  \item[hsalRA.exe:]
    The HybridSal relational abstracter, combined with
    the infinite bounded model checkers.
    \\
    The SAL tool is required to be installed separately.
    SAL executables are available from the SAL website.
    On Windows, SAL runs only under cygwin.
    \\
    The hybridsal tool is part of META release.
\end{description}

\section{A Simple Exercise}

We describe a simple design exercise using
HybridSal.  It consists of the following steps.
\begin{itemize}
  \item Building/editing the cyber model and
    annotating it with LTL properties
  \item Converting the Matlab model into the
    cyber composition language
  \item Creating a HybridSal representation of
    the model and the properties
  \item Verifying the HybridSal model
\end{itemize}

\subsection{Building/editing the cyber model}
We start with controllers designed by VU.
The Matlab files for these controllers are:
\begin{itemize}
  \item Torque\_Converter\_control.mdl
  \item TorqueReductionSignal.mdl
  \item SimplifiedShiftController.mdl
\end{itemize}

These files can be opened in Simulink and edited.
If the LTL-extension is installed, then LTL properties
can be attached to these models directly by right clicking
on the model.

Models (and the properties) are saved as .mdl files
using Matlab.

\subsection{Converting Matlab to Cyber Composition language}

Using the {\tt{MDL2MGACyber.exe}} tool, an XML representation
of the models is generated.
Corresponding to the above mdl files,
we get the file 

\begin{center}
   {\tt{SimplifiedControllerCyberWithProp.xml}}
\end{center}

This file has all the three cyber-components and 
the attached properties.

\subsection{Converting CyberComposition XML to HybridSal}

The python script, {\tt{cybercomposition2hsal.py}},
converts the XML into HybridSal.
It is run as follows:

\begin{quote}
  {\tt{
      python cybercomposition2hsal.py SimplifiedControllersCyberWithProp.xml
  }}
\end{quote}

Running the above script will generate a file called
\begin{center}
{\tt{SimplifiedControllersCyberWithProp.hsal}}.
\end{center}

Users can open this file and see the formal representation
of the three cyber-components and the LTL properties.

\subsection{Verifying the HybridSal model}

The HybridSal file can now be model checked
using the following command:
\begin{quote}
  {\tt{
      hsalRA.exe SimplifiedControllersCyberWithProp p1
  }}
\end{quote}
where {\tt{p1}} is the name of the LTL property being verified.

Running this command performs the following actions:
\begin{itemize}
  \item A relational abstraction of the HybridSal model
    is first constructed. The result is stored in a file
    with the same filename, but with extension ``.sal''.
    \\
    In the above example, the tool will create a new file
    called 
    \begin{center}
      {\tt{SimplifiedControllersCyberWithProp.sal}}
    \end{center}
  \item The SAL model is model-checked using an infinite-state
    model checker.
    \\
    The output is either a counter-example for the property,
    or a statement that no counter-example was found.
\end{itemize}

These two steps can be independently performed on the command-line
as follows:
\begin{quote}
  {\tt{
      python HSalRelAbsCons.py SimplifiedControllersCyberWithProp.hsal
      \\
      sal-inf-bmc -d 10 -v 3 SimplifiedControllersCyberWithProp p1
  }}
\end{quote}

Design/property changes can
be made using Matlab on the Simulink/Stateflow models.
The verification process can then be repeated.
Alternatively, it is also possible to edit the
intermediate HybridSal or Sal files directly --
this can save time -- but the edits the not carried
back to the Matlab models.

\section{Remarks}

In the above tutorial, only the cyber components
were being analyzed.
The plant model is completely abstracted.
Hence, the analysis can be coarse.
But, the analysis can still be useful, especially
if care is taken in specifying the LTL properties.
In particular, LTL properties should constrain the
inputs of the controllers, and then check that the
response of the controller is appropriate for that
scenario.  The predefined LTL templates may not be
sufficient for this purpose.

The properties {\tt{p1}} and {\tt{p2}} in the
HybridSal files were included to illustrate the
utility. 
Property {\tt{p1}} says that if  the input
{\tt{shift\_requested}} is consistently set to $1$,
then after a few steps,  the output 
{\tt{gear\_selected}} of the controller is $1$.
This property was found to be invalid.
Property {\tt{p2}} says that if  the input
{\tt{shift\_requested}} is consistently set to $1$,
then after a few steps,  the output 
{\tt{gear\_selected}} of the controller is at most $2$.
No counter-examples were found for this property.
However, it is possible to fix the controller logic
to make property {\tt{p1}} true.
Failure of {\tt{p1}} was possibly a bug in the designed
controller.

\end{document}

