\documentclass{article}

\usepackage{amssymb}
\usepackage{url}
\newtheorem{definition}{Definition}

\begin{document}

\title{HybridSAL Relational Abstractor: A New Abstractor for Hybrid Systems}
\author{Ashish Tiwari}

\maketitle

\section{Detailed Example and the HybridSal Relational Abstractor Tool}

Consider a simple 2-dimensional continuous system defined by
\begin{eqnarray*}
\frac{dx}{dt} & = & -y + x
\\
\frac{dy}{dt} & = & -y - x
\end{eqnarray*}
Assume we are given the initial condition $x = 1, y = 2$
and the invariant that $y$ is always non-negative.
We wish to prove that $x$ always remains non-negative.

\begin{figure}[t]
\begin{tt}
\begin{verbatim}
Linear1: CONTEXT = 
BEGIN

control: MODULE =
BEGIN
 LOCAL x,y:REAL
 LOCAL xdot,ydot:REAL
 INITIALIZATION
  x = 1; y = 2
 TRANSITION
 [
 y >= 0 AND y' >= 0 --> 
  xdot' = -y + x ;
  ydot' = -y - x
 ]
END;

% proved using sal-inf-bmc -i -d 2 Linear1 helper
helper: LEMMA
 control |- G(0.9239 * x >= 0.3827 * y);

% proved using sal-inf-bmc -i -d 2 -l helper Linear1 correct
correct : THEOREM
 control |- G(x >= 0);
END
\end{verbatim}
\end{tt}
\caption{HybridSAL file describing a simple two-dimensional
continuous dynamical system, along with two safety
properties of that system.}\label{fig:hsal-ex}
\end{figure}

This example can be encoded in HybridSAL as shown in
Figure~\ref{fig:hsal-ex}.  The HybridSAL syntax is 
almost identical to the syntax of SAL~\cite{sal}, but for a few
modifications that enable encoding of continuous dynamical
systems. The key changes in HybridSAL are:
\begin{itemize}
\item
Variables whose name ends in {\em{dot}} denote the derivative.
In Figure~\ref{fig:hsal-ex}, there are two state variables
$x,y$, and their derivatives are denoted by variables named
$xdot,ydot$ respectively.  These special dot variables can only
be used as left-hand sides of simple definitions (equations)
that appear in guarded commands.  Thus, the equation
$xdot' = -y+x$ denotes the differential equation
$dx/dt = -y+x$.
\item
The guard of the guarded command that encodes the system of
differential equation denotes the state invariant; that is,
the system is forced to remain inside the invariant set while
evolving as per the differential equations.
This meaning is consistent with the usual semantics of guards
in SAL. 
In Figure~\ref{fig:hsal-ex}, the guard $y \geq 0\wedge y'\geq 0$
says that $y\geq 0$ is the mode invariant for the only
mode in the system.
\end{itemize}
The definition of contexts, modules, and properties (theorems)
are exactly as in the SAL language~\cite{sal}.

The user creates a HybridSAL model, similar to the one shown 
in Figure~\ref{fig:hsal-ex}, using a text editor.
Following the SAL convention for naming files, the HybridSAL file containing
the model in Figure~\ref{fig:hsal-ex} is stored as file
{\tt{Linear1.hsal}}.
Thereafter, to prove the two properties contained in 
{\tt{Linear1.hsal}}, the user executes the
following commands.
\begin{description}
\item[{\tt{bin/hsal2hasal examples/Linear1.hsal}}].
This command is run from
the root directory of the HybridSAL relational abstraction tool.
It will create a set of files in the subdirectory {\tt{examples/}},
including the file {\tt{Linear1.sal}} that contains the
relational abstraction of the original model.
The process of going from {\tt{Linear1.hsal}} to 
{\tt{Linear1.sal}} involves the following stages
(that are all performed in a single run of the above command):
\[
\begin{array}{rclclclcl}
{\tt{Linear1.hsal}} \longrightarrow
{\tt{Linear1.hxml}} \longrightarrow
{\tt{Linear1.haxml}} \longrightarrow 
{\tt{Linear1.xml}} \longrightarrow
{\tt{Linear1.sal}}
\end{array}
\]
The hsal2hxml converter is in the subdirectory 
{\tt{hybridsal2xml}}.  It is simply the HybridSAL
parser that parses a files and outputs it in {\tt{.hxml}} 
format.  The program {\tt{bin/hsal2hasal}} can take as input
either a {\tt{.hsal}} file or a {\tt{.hxml}} file.
It creates {\tt{.hasal}} and {\tt{.haxml}} -- which is 
a file in {\em{extended HybridSAL}} syntax -- that contains
the original model as well as its relational abstraction.
It is an intermediate file that is useful only for debugging
purposes at the moment.
In the last stage, the final {\tt{.xml}} and {\tt{.sal}} files 
are easily extracted
from the {\tt{.haxml}} files.
\item[{\tt{sal-inf-bmc -i -d 2 Linear1 correct}}].
Once the relational abstraction has been created, it can be 
model checked.  Note that the relational abstraction (in
file {\tt{Linear1.sal}}) is an {\em{infinite}} state system.
Hence, we can not use finite state model checkers. 
We can, however, use the SAL infinite bounded model checker
({\tt{sal-inf-bmc}}) and 
the k-induction prover
({\tt{sal-inf-bmc -i}}).
The k-induction prover can sometimes fail to prove a correct
assertion because the assertion is not inductive. In such a
case, auxiliary lemmas may be needed to complete a proof.
In the running example, we need a helper lemma.
Using the lemma {\tt{helper}}, the property {\tt{correct}}
can be proved using the command:
\\
 {\tt{sal-inf-bmc -i -d 2 -l helper Linear1 correct}}
\\
The lemma {\tt{helper}} can itself be proved using $k$-induction
as:
\\
 {\tt{sal-inf-bmc -i -d 2 Linear1 helper}}
\end{description}

This completes the discussion of the application of  the
HybridSAL relational abstraction tool on the running example.
For more complex examples, including examples of hybrid systems,
the reader is referred to the {\tt{examples/}} subdirectory in 
the tool.
We note a few points here.
\begin{description}
\item[Initialization]
The initial state of the system need not be a single point.
It can be a region of the state space.  For example,
in Figure~\ref{fig:hsal-ex}, 
the initialization section can be replaced by
the initialization 
\begin{tt}
 \begin{verbatim}
 INITIALIZATION 
  x IN {z:REAL|0 <= z AND z <= 1};
  y IN {z:REAL|2 <= z AND z <= 3};
 \end{verbatim}
\end{tt}
The new model can again be verified using the same set of commands
given above.
\item[Composition]
The model need not be a single module, and it can be a composition of 
modules.  Modules can be purely discrete -- they need not all have
dynamics given by differential equations.  For example, the example
in the File {\tt{TGC.hsal}} describes a model of the train-gate-controller
in HybridSAL that is a composition of five modules.  Only one of the
five has continuous differential equations in the dynamics.
\end{description}

Apart from the syntax for writing differential equations,
the HybridSAL input language supports two additional features
that are not part of the SAL language~\cite{sal}.
These features are:
\begin{description}
\item[Invariant]
 Apart from {\tt{INITIALIZATION}} and {\tt{TRANSITION}} blocks,
 each basemodule can also have an {\tt{INVARIANT}} block.
 The invariant block contains a formula that is an (assumed)
 invariant of the system.
 
 In our running example, we can add the Invariant block:
 \begin{tt}
 \begin{verbatim}
  INVARIANT y >= 0
 \end{verbatim}
 \end{tt}
 in the basemodule, for example, just before/after the 
 {\tt{INITIALIZATION}} block.
 Then, we could replace the guard
 $y \geq 0 \wedge y'\geq 0$ by the new guard $\mathit{True}$
 in the (only) transition.
 If $\phi$ is declared as the invariant, then it has the
 effect of adding the formula $\phi \wedge \phi'$ in the guards
 of {\em{all}} transitions.  This is performed as a preprocessing
 step by the HSal Relational Abstractor.

\item[INITFORMULA]
 Instead of {\tt{INITIALIZATION}} block, a HybridSAL input file
 can contain a {\tt{INITFORMULA}} block that has a formula
 as the initialization predicate.
 
 In our running example, the initialization block shown above
 can be replaced by
 \begin{tt}
 \begin{verbatim}
  INITFORMULA 0 <= x AND x <= 1 AND 2 <= y AND y <= 3
 \end{verbatim}
 \end{tt}
 without changing the meaning of the HybridSAL model.

 The {\tt{INITFORMULA}} block is also handled during the
 preprocessing phase.
 The preprocessing phase also handles {\em{defined constants}}.
\end{description}

\section{HSal Relational Abstractor: Flags}

The HybridSAL Relational Abstractor tool accepts the following options/flags.
\begin{description}
\item[-n, --nonlinear]
 With this option, the tool creates a more precise relational abstraction,
 but it may be nonlinear.  Even when the input model is a linear
 continuous dynamical system, the output could be a nonlinear discrete
 time system.
 
 Currently, the SAL model checker can not analyze nonlinear discrete time
 systems.  Hence, this flag is useful only if using other backend tools
 that can handle discrete time nonlinear systems.

 By default, this option is {\em{not}} turned on, and the tool
 creates linear relational abstractions that can be analyzed by SAL
 infinite bounded model checker.

\item[-c, --copyguard]
 With this option, the tool explicitly handles the guard in the 
 continuous dynamics as state invariants.
 Recall the HybridSAL code for the differential equations
 $dx/dt = -y+x, dy/dt = -y-x$.
 \begin{tt}
 \begin{verbatim}
 [
 y >= 0 AND y' >= 0 --> 
  xdot' = -y + x ;
  ydot' = -y - x
 ]
 \end{verbatim}
 \end{tt}
 Here the guard $y \geq 0 \wedge y'\geq 0$ says that $y$ should be
 nonnegative both {\em{before}} and {\em{after}} the transition.
 In other words, $y\geq 0$ is the mode invariant.
 We could have written the same dynamics as follows:
 \begin{tt}
 \begin{verbatim}
 [
 y >= 0 --> 
  xdot' = -y + x ;
  ydot' = -y - x
 ]
 \end{verbatim}
 \end{tt}
 The new HybridSAL file, when processed with the flag {\tt{-c}},
 produces the same output as the original file would produce 
 without the {\tt{-c}} flag.
 Thus, the {\tt{-c}} flag causes copying of the guard 
 of the continuous transitions, but
 with variables replaced by their prime forms.

 By default, this option is {\em{not}} turned on, and the tool
 assumes that the input HybridSAL file already 
 contains guards on prime variables.

\item[-o, --opt]
 This flag turns on some optimizations in the relational
 abstractor.
 Currently, this flag causes the tool to construct a 
 relational abstraction that assumes that a certain amount
 of time is always spent in each mode.  This is an
 unsound assumption.  Hence, the output of the relational
 abstractor can be unsound when the {\tt{-o}} flag is used.
 
 The {\tt{-o}} flag is useful if the (sound) relational abstraction
 is too large to be analyzable (in reasonable time) by the model checker.
 In that case, the user could try to create a simpler, but unsound,
 abstraction using the {\tt{-o}} flag and model checking it.
 
 By default, this option is {\em{not}} turned on.

\item[-t $\langle$timestep$\rangle$, --time $\langle$timestep$\rangle$]
 This flag is still under development.
 It constructs  a timed relational abstraction of the given system.
 This is useful in cases where all discrete mode switches are
 {\em{time triggered}}.  The value of {\tt{timestep}} should be
 a real number, which is interpreted as the inverse of the
 sampling frequency.  It is important to choose the
 timestep carefully.
\end{description}

The tool is expected to include more options in the future
as new extensions and features are implemented.

\section{HSal Relational Abstractor: Background}

Hybrid dynamical systems are formal models of complex systems 
that have both discrete and continuous behavior.  
It is well-known that the problem of verifying hybrid systems for
properties such as safety and stability is quite
hard, both in theory and in practice.  There are no automated,
scalable and compositional tools and techniques for formal 
verification of hybrid systems.

HybridSAL is a framework for modeling and analyzing hybrid systems.
HybridSAL is built as an extension of SAL (Symbolic Analysis Laboratory).
SAL consists of a language and a suite of tools for modeling and 
analyzing discrete state transition systems.  HybridSAL extends SAL 
by allowing specification of continuous dynamics in the form of 
differential equations.
Thus, HybridSAL can be used to model hybrid systems.
These models can be abstracted into discrete  finite state transition
systems using the HybridSAL abstractor.  The abstracted system
is output in the SAL language, and hence SAL (symbolic) model 
checkers can be used to model check the abstraction.
While HybridSAL can be used to verify intricate hybrid system models,
it is based on constructing qualitative abstractions -- which can
be very coarse at times.  Furthermore, the HybridSAL abstractor is
not compositional, and can not abstract modules independently
separately without being very coarse.

To alleviate the shortcomings of the HybridSAL abstractor,
we developed the concept of relational abstractions of hybrid
systems.  A relational abstraction transforms a given hybrid system into
a purely discrete transition system by summarizing the effect
of the continuous evolution using relations.   The state space
of system and its discrete transitions are left unchanged. 
However, the differential equations describing the
continuous dynamics (in each mode) are replaced by a 
relation between the initial values of the variables and 
final values of the variables.
The abstract discrete system is an infinite-state system that can be
analyzed using standard techniques for verifying systems such
as $k$-induction and bounded model checking.

Relational abstractions can be constructed compositionally by 
abstracting each mode separately.  Abstraction and compositionality
are crucial for achieving scalability of verification.
We have also developed techniques for constructing good quality 
relational abstractions.  The details are technical and can be 
found in papers~\cite{ST11:CAV,Tiwari03:HSCC}.  The HybridSAL verification framework has 
been extended by an implementation of relational abstraction.


\subsection{Qualitative versus Relational Abstraction}

Qualitative and predicate abstraction are techniques for abstracting
a system that work by simplifying the state space of the system.
Specifically they reduce the state space of the system to a finite
set of (qualitative) states defined by certain (qualitative) predicates.
In contrast, relational abstraction does not simplify the state space
but only simplifies the presentation of the dynamics by replacing
hard-to-analyze differential equations by discrete transitions.
In principle, predicate and qualitative abstraction can be used on
a relational abstraction of a system to further approximate the
system, if needed.  The original HybridSAL tool implements qualitative
abstraction.  The new HybridSAL Relational Abstractor implements
relational abstraction. 


\section{HSal Relational Abstractor: Technical Background}

We extended the HybridSAL tool by implementing a 
relational abstractor for HybridSAL models.
The input to the tool is a specification of a hybrid system 
in the HybridSAL language.  

The {\bf{verification workflow}} for using the HybridSAL relatioal
abstractor is as follows:
\begin{enumerate}
\item  Create a model of a hybrid system in HybridSAL:
 The user creates such a model in a file, say in 
 file filename.hsal, using any text editor.
\item
 Construct a relational abstraction:
 The user runs the HybridSAL relational abstractor to create a discrete,
 but infinite-state relational abstraction of the original model.
 The relational abstraction is output in the SAL language as
 filename.sal.
\item
 Analyze the SAL model:
 The user runs the SAL infinite bounded model checker or k-induction prover
 to analyze the SAL model in filename.sal
\end{enumerate}

We now discuss the implementation of the HybridSAL relational abstractor.
We first define relational abstraction of a continuous dynamical
system. 

\begin{definition}[Relational Abstraction Without Time]\label{def-ra}
Consider a continuous dynamical system 
$\frac{d\vec{x}}{dt} = f(\vec{x})$, where $\vec{x}$ is a $n\times 1$
vector of real-valued variables.
The relation $R \subseteq \mathbb{R}^{2n}$ is a relational abstraction 
of this continuous system if  for all time
trajectories $\tau: [0,T) \mapsto \mathbb{R}^n$ of this system,
it is the case that
$ (\forall\ t \in [0,T))\ (\tau(0),\tau(t)) \in R$.
\end{definition}
Thus, a relational abstraction $R$
captures all pairs of states $(\vec{x}_0,\vec{x})$ such that it is possible to
reach $\vec{x}$ from $\vec{x}_)$ in a finite amount of time by evolving
according to the continuous dynamics.
A relational abstraction of a hybrid system is constructed by 
replacing each constituent continuous system by its relational 
abstraction and keeping the discrete transitions unchanged.

We briefly describe the techniques implemented in the
HybridSAL relational abstractor.
Currently, HybridSAL relational abstractor can only construct
relational abstractions of hybrid systems in which all
continuous dynamics are given by linear ordinary differential
equations.
For linear systems, such as
$d\vec{x}/dt = A\vec{x}$, whenever $A$ has real eigenvalues, useful
relational abstractions can be generated using the eigenvectors of
${A'}$ corresponding to those real
eigenvalues~\cite{Tiwari03:HSCC}.  Here, ${A'}$ denotes the
transpose of matrix $A$.  Specifically, if $\vec{c}$ is such that
${A'}\vec{c} = \lambda \vec{c}$, then by simple algebraic
manipulation, we obtain $ \frac{d}{dt} (c_1 x_1 + \ldots + c_n x_n) =
\lambda (c_1 x_1 + \ldots + c_n x_n)$ where $\vec{c} :=
[c_1;\ldots;c_n]$ and $\vec{x} := [x_1;\ldots;x_n]$.  Let $p$ denote
the linear expression $c_1 x_1 + \ldots + c_n x_n$ and let $p_0$
denote the linear expression $c_1 x_{10} + \ldots + c_n x_{n0}$.  Here,
$x_{i0}$ denotes the old value of $x_i$.  If $\lambda < 0$, then we
know that the value of $p$ approaches zero monotonically.  Consequently, we
get the relational abstraction 
$ (p_0 \leq p < 0) \vee
  (p_0 \geq p > 0) \vee (p = p_0 = 0) $. Similarly, we can
write the relational invariants for the case when $\lambda > 0$ and
$\lambda = 0$.
This key idea also generalizes to the case when 
$\frac{d\vec{x}}{dt} = A\vec{x}+\vec{b}$.
We call such relational abstractions eigen-invariants.

When $A$ has complex eigenvalues, we get other kinds of
relational abstractions; for details, the reader is
referred to~\cite{Tiwari03:HSCC}.  

\begin{table}[t]
{\footnotesize \begin{center}
\begin{tabular}{|c||c|c|c||c|c|c||c|c|c|}
\hline
Benchmark & 
\multicolumn{3}{c||}{Affine Invs} & 
\multicolumn{3}{c||}{Affine+Eigen Invs} & 
\multicolumn{3}{c|}{Affine+Eigen+Box Invs}
\\ \hline
 & depth & status & time(s)
 & depth & status & time(s)
 & depth & status & time(s)
\\ \hline\hline
nav01  
& 4 & F & 0.63
& 4 & F & 0.88
& 4 & F & 1.91
\\ 
nav01  
& 5 & P & 0.75
& 5 & P & 0.91
& 5 & P & 1.36
\\ \hline
nav02
& 4 & F & 0.64
& 4 & F & 0.87
& 4 & F & 1.8
\\ 
nav02
& 5 & P & 0.68
& 5 & P & 1.04
& 5 & P & 3.33
\\ \hline
nav03
& 4 & F & 0.60
& 4 & F & 0.91
& 4 & F & 1.72
\\ 
nav03
& 5 & P & 0.67
& 5 & P & 1.05
& 5 & P & 2.7
\\ \hline
nav04
& 3 & CE & 0.49
& 8 & F & 3.21
& 8 & F & 34.883
\\ 
nav04
& & &  
& 4 & P & 0.75+0.99
& 4 & P & 0.98+2.21
\\ \hline
nav05
& 2 & CE & 0.47
& 8 & F & 3.85
%& 5 & F & 8.45
& 8 & F & 37.31
\\ 
nav05
& & &
& 8 & P & 2.15+2.50
& 8 & P & 5.38+11.05
\\ \hline
nav06
& 4 & CE & 0.61
& 8 & F & 18.01
& 8 & F & 494.5
\\ 
nav06*
& 4 & CE & 1.03
& 8 & P & 21.80+7.42  
& 8 & P & 40.22+35.08  
\\ \hline
nav07
& 5 & CE & 0.66
& - & - & - 
& 5 & F & 69.9 
\\ 
nav07
& & &
& - & - &- 
& 6 & P & 6.25 
\\ \hline
nav08
& 4 & CE & 0.52
& - & - &- 
& 6 & CE & 0.95
\\ \hline
nav09
& 4 & CE & 0.57
& 4 & CE & 1.45
& 4 & CE & 19.87
\\ \hline
nav10
& 3 & CE & 0.44
& 3 & CE & 0.99
& 3 & CE & 0.95
\\ \hline
\end{tabular}
\end{center}
}
\caption{Comparison of various abstractions over the NAV
  benchmarks. All experiments were performed on an Intel Xeon E5630 2.53GHz
  single-core processor (x86\_64 arch) with 4GB RAM running Ubuntu
  Linux 2.6.32-26.  Legend --- \textbf{depth:}
  $k$-induction depth, \textbf{time:} time taken by verifier,
  \textbf{status:}  \textbf{P}: Proved
  Property, \textbf{CE:} $k$-induction base case fails and
  counterexample is produced, \textbf{F}: inductive step fails, no
  proofs or counterexample.  \textbf{Note:} Relational
  eigeninvariants are inapplicable for nav07, nav08 (indicated by -).  
$k$-induction timings  reported as $t_1+t_2$ indicate that an auxiliary lemma was used. $t_1$ is the time to prove the
  property, and $t_2$  to discharge the
 lemma.
}\label{table-exp}
%\vspace*{-1cm}
\end{table}

We evaluate the HybridSAL relational abstractor over  the navigation
benchmarks~\cite{benchmarks}.
The navigation benchmarks model a vehicle moving in a
2-dimensional rectangular space $[0,m-1]\times [0,n-1]$.
This space is partitioned in $m\times n$ cells.  Let
$x,y$ denote the position of the vehicle and $v_x,v_y$ denote
its velocity.  Then the dynamics of the vehicle in any particular
cell are given by the ODEs:
\[
\begin{array}{rcl@{\quad}rcl}
 \frac{dx}{dt} &  = & v_x
 &
 \frac{dv_x}{dt} &  = & a_{11} (v_x - b) + a_{12} (v_y - c)
\\
 \frac{dy}{dt} &  = & v_y
 &
 \frac{dv_y}{dt} &  = & a_{21} (v_x - b) + a_{22} (v_y - c)
\end{array}
\]
where the matrix $A := [a_{11},a_{12};a_{21},a_{22}]$ and the
direction $(b,c)$ are parameters that can potentially vary 
(for each of the cells)%
\footnote{The matrix $A$ is Hurwitz:
the dynamics for $(v_x,v_y)$ asymptotically converge to
$(b,c)$.}.

Every benchmark in the suite is
specified by fixing the matrix $A$, the number of cells $m \times n$,
the direction $(b,c)$ in each cell, and initial intervals for each of
the four state variables $x,y,v_x,v_y$.   Our experiments
focus on proving the unreachability of a distinct cell marked $B$ for each benchmark instance~\cite{benchmarks}.

In our experiments, we verify the safety property for the navigation
benchmarks using $k$-induction over the relational abstraction.  We
use the SAL infinite bounded model checker, with the $k$-induction
flag turned on ({\tt{sal-inf-bmc -i}}), which uses the SMT solver
Yices in the back end.  Table~\ref{table-exp} reports the results. For
each benchmark, we report the depth used for performing $k$-induction
(under ``depth''), the output of $k$-induction (under ``status''), and
the time it took (under ``time'').  There are three possible outputs:
(a) the base case of $k$-induction fails and a counterexample is
found (denoted by ``CE''), (b) the base case is proved, but the
induction step fails; i.e., no counterexample is found, but no proof
is found either (denoted by ``F''), (c) the base case and the induction
step are successfully proved (denoted by ``P''). Since we perform
$k$-induction on an abstraction, the counterexamples may be spurious,
but the proofs are not.  As Table~\ref{table-exp} indicates,
relational abstractions are sufficient to establish safety of the
benchmarks nav01--nav05, nav06*, and nav07.  The system nav06* is the
same as nav06 but with a slightly smaller set of initial states.
However, the proof fails on nav06 and nav08--nav10.  There are two
reasons for failure: (a) poor quality of abstraction, which is
reflected in entries ``CE'' in Table~\ref{table-exp},  and (b) inability
to find suitable k-inductive lemmas. This happens in the case of
nav06, where the proof fails without yielding a counterexample.  
We employed three kinds of relational abstractions for each mode:
affine, eigen, and box.  Table~\ref{table-exp} also shows
performance of each of these techniques.
 
For further details on relational abstraction, and for downloading
the HybridSAL relational abstractor and documentation, the reader
is referred to the website
{{\url{http://www.csl.sri.com/~tiwari/relational-abstraction/}}}.


\section{Conclusion}

Compositional verification of complex systems is a
challenging problem.  We developed a new approach
that enables compositional analysis of continuous and
hybrid dynamical systems.  It is based on computing 
relational abstractions of continuous dynamics of
a complex system.

Relational abstraction replaces the differential equations 
in the system description by sound abstract discrete 
transitions, thus enabling application of discrete verification 
tools. The HybridSAL relational abstractor automatically 
computes such abstractions for linear hybrid systems. 
The extension to nonlinear dynamics is left for future
work.

\bibliography{hsal-ra-report}
\bibliographystyle{plain}

%\begin{verbatim}
%[1] Sriram Sankaranarayanan and Ashish Tiwari, 
   %"Relational Abstractions for Continuous and Hybrid Systems", In CAV 2011: 686-702. 
%
%[2] Ashish Tiwari, "Approximate Reachability for Linear Systems", In the 
 %Proceedings of Hybrid Systems: Computation and Control, HSCC 2003. 
%
%[3] http://www.csl.sri.com/users/tiwari/existsforall/
%
%[4] Ashish Tiwari, "Compositionally analyzing a PI controller family", 
  %In CDC 2011. To appear. 
%
%[5] Thomas Sturm and Ashish Tiwari, 
  %"Verification and synthesis using real quantifier elimination". In ISSAC 2011. 
%[Tiwari/03/Approximate] Ashish Tiwari,
 %"Approximate reachability for linear systems".
 %In HSCC 2003, pp. 514--525.
%
%[Fehnker+Ivancic/2004/Benchmarks] Ansgar Fehnker and Franjo Ivancic,
 %"Benchmarks for Hybrid Systems Verification". In HSCC 2004, pp. 326-341.
%\end{verbatim}

\end{document}
