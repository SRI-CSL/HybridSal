\documentclass{llncs}
%\usepackage{sriram-macros}
\usepackage{url}
\usepackage{amsmath,amssymb,amsfonts}
%\usepackage{electComp}
\pagestyle{plain}

\title{HybridSAL Relational Abstracter}
%  \thanks{Tiwari's work supported in part by DARPA under Contract No. FA8650-10-C-7078,
%  NSF grants CSR-0917398 and SHF:CSR-1017483.}
\author{Ashish Tiwari}
\institute{SRI International, Menlo Park, CA.\ \url{ashish.tiwari@sri.com}}

\newif\iftrversion\trversionfalse
\def\linODEabs{\tt{linODEabs}}

\begin{document}

\maketitle

\begin{abstract}
  In this paper, we present the HybridSAL relational abstracter -- 
  a tool for verifying continuous and hybrid dynamical
  systems.  The input to the tool is a model of a hybrid dynamical
  system and a safety property.  The output of the tool is a discrete
  state transition system and a safety property.  The correctness
  guarantee provided by the tool is that if the output property 
  holds for the output discrete system, then the input property
  holds for the input hybrid system.  The input is in HybridSal
  input language and the output is in SAL syntax. The SAL model
  can be verified using the SAL tool suite.  This paper describes 
  the HybridSAL relational abstracter -- the algorithms it implements, 
  its input, its strength and weaknesses, and its use for verification
  using the SAL infinite bounded model checker and k-induction prover.
\end{abstract}

\def\XX{\mathbb{X}}
\def\QQ{\mathbb{Q}}
\def\YY{\mathbb{Y}}
\def\RR{\mathbb{R}}
\def\ra{\rightarrow}
%\def\bra{{\crm{\ra}}}  % make it \ra_a
\def\bra{{\stackrel{a}{\ra}}}  % make it \ra_a
%\def\rra{{\cem{\ra}}}  % make it \ra_c
\def\rra{{\stackrel{c}{\ra}}}  % make it \ra_c

\section{Introduction}

A dynamical system $(\XX, \bra)$  with state space $\XX$ and
transition relation $\bra\subseteq \XX\times \XX$ 
is a {\em{relational abstraction}} of another 
dynamical system $(\XX, \rra)$ if
the two systems have the same state space and
$\rra \subseteq \bra$.
Since a relational abstraction contains all the behaviors of the
concrete system, it can be used to perform safety verification.

HybridSAL relational abstracter is a tool that computes a relational
abstraction of a hybrid system as described by
Sankaranarayanan and Tiwari~\cite{ST:CAV2010}.
A hybrid system $(\XX, \ra)$ is a dynamical system with 
\\
(a) state space $\XX := \QQ\times\YY$, where $\QQ$ is a finite set and
$\YY := \RR^n$ is the $n$-dimensional real space,
and 
\\
(b)
transition relation $\ra := \ra_{cont}\cup\ra_{disc}$, where
$\ra_{disc}$ is defined in the usual way using guards and
assignments, but $\ra_{cont}$ is defined by a system of
{\em{ordinary differential equation}} and a {\em{mode invariant}}.
One of the key steps in defining the (concrete)
semantics of hybrid systems
is relating a system of differential equation
$\frac{d\vec{y}}{dt} = f(\vec{y})$ with mode invariant
$\phi(\vec{y})$ to a binary relation 
over $\RR^n$, where $\vec{y}$ is a $n$-dimensional vector of
real-valued variables.  Specifically, the semantics of
such a system of differential equations is defined as:
\begin{eqnarray}
\vec{y}_0 \ra_{cont} \vec{y}_1 & &
\mbox{if}\;
\mbox{there is a $t_1\in\RR^{\geq 0}$ and 
a function $F$ from $[0,t_1]$ to $\RR^n$ s.t.} \nonumber %\label{eqn:sem}
\\ & &
\vec{y}_0 = F(0),
\vec{y}_1 = F(t_1), \mbox{ and } \nonumber
\\ & &
\forall{t\in [0,t_1]}:
\left(\frac{dF(t)}{dt} = f(F(t)) \wedge \phi(F(t))\right)\label{eqn:sem}
%
%\vec{y}_0 = F(0),
%\vec{y}_1 = F(t_1), \mbox{ and }
%\frac{dF(t)}{dt} = f(F(t)) \nonumber
%\\ & &
%\mbox{for some $t_1\in\RR^{\geq 0}$ and 
%some function $F$ from $[0,t_1]$ to $\RR^n$} \label{eqn:sem}
\end{eqnarray}
The concrete semantics is defined using the ``solution''
$F$ of the system of differential equations.  As a result,
it is difficult to work directly with the concrete semantics.

The relational abstraction of a hybrid system 
$(\XX, \rra_{cont}\cup\rra_{disc})$ 
is a discrete state transition system $(\XX, \bra)$ such that
$\bra = \bra_{cont}\cup\rra_{disc}$, where 
$\rra_{cont} \subseteq \bra_{cont}$.
In other words, the discrete transitions of the hybrid system
are left untouched by the relational abstraction, and only the
transitions defined by differential equations are abstracted.

The HybridSal relational abstracter tool computes such a relational
abstraction for an input hybrid system. 
In this paper, we describe the core algorithms implemented in the
tool, the tool interfaces and the examples and case studies that
are part of the tool distribution.

\section{Relational Abstraction of Linear Systems}

Given a system of linear ordinary differential equation,
$\frac{d\vec{x}}{dt} = A\vec{x}+\vec{b}$, we describe the
algorithm used to compute the abstract transition relation
$\bra$ of the concrete transition relation $\rra$ defined
by the differential equations.

The algorithm is described in Figure~\ref{fig:algo}.  
The input is a pair $(A,b)$, where
$A$ is a $(n\times n)$ matrix of rational numbers
and $\vec{b}$ is a $(n\times 1)$ vector of rational numbers.
The pair represents a system of differential equations
$\frac{d\vec{x}}{dt} = A\vec{x}+\vec{b}$.
The output is a formula $\phi$ over the variables
$\vec{x},\vec{x}'$ that represents the relational
abstraction of $\frac{d\vec{x}}{dt} = A\vec{x}+\vec{b}$.
The key idea in the algorithm below is to use 
the eigenstructure of the matrix $A$ to generate the
relational abstraction.

\begin{figure}[htb!]
{\linODEabs}$(A,b)$:
{\em{Input}}: a pair $(A,b)$, where
$A\in\RR^{n\times n}, b\in\RR^{n\times 1}$.
\\
{\em{Output}}: a formula $\phi$ over the
variables $\vec{x},\vec{x}'$ 
\begin{enumerate}
\item
 identify variables that have a constant rate of change
\\
 suppose $x_1, \ldots, x_k$ are all such variables and
 $\frac{dx_i}{dt} = b_i$ where $b_i\in\RR\;\;\forall{i}$
\\
 %let $\phi_1$ be $\bigwedge_{i=2}^{k} \frac{x_i'-x_i}{b_i} = \frac{x_1'-x_1}{b_1}$
 let $E$ be $\{\frac{x_i'-x_i}{b_i} \mid i=1,\ldots,k\}$
\item
 partition the variables $\vec{x}$ into $\vec{y}$ and $\vec{z}$ such that
 the system of differential equations can be rewritten as
 \begin{eqnarray*}
  \left[ \begin{array}{c} \frac{d\vec{y}}{dt} \\ \frac{d\vec{z}}{dt} \end{array} \right]
  & = &
  \left[ \begin{array}{cc} A_1 & A_2 \\ 0 & 0 \end{array} \right]
  \left[ \begin{array}{c} \vec{y} \\ \vec{z} \end{array} \right] +
  \left[ \begin{array}{c} \vec{b_1} \\ \vec{b_2} \end{array} \right]
 \end{eqnarray*}
 where 
  $A_1 \in \RR^{n_1\times n_1}$,
  $A_2 \in \RR^{n_1\times n_2}$,
  $\vec{b_1} \in \RR^{n_1\times 1}$, 
  $\vec{b_2} \in \RR^{n_2\times 1}$, and $n = n_1 + n_2$
\item set $\phi$ to be $\mathit{True}$
\item\label{loop1head}
 let $\vec{c}$ be a real left eigenvector of matrix $A_1$ and
 let $\lambda$ be the corresponding real eigenvalue, that is,
 \begin{eqnarray*}
  \vec{c}^T A_1 & = & \lambda \vec{c}^T
 \end{eqnarray*}
\item
 if $\lambda == 0\wedge c^TA_2 == 0$: set $E := E\cup \{ \frac{\vec{c}^T(\vec{y}'-\vec{y})}{\vec{c}^T\vec{b_1}} \}$; else: $E := E$
\item
 if $\lambda \neq 0$:
 define vector $\vec{d}$ and real number $e$ as follows:
 \begin{eqnarray*}
  \vec{d}^T  \; = \; \frac{\vec{c}^T A_2}{\lambda}
   & \qquad &
  e  \; = \; \frac{\vec{c}^T \vec{b_1} + \vec{d}^T \vec{b_2}}{\lambda}
 \end{eqnarray*}
 let $p(\vec{x})$ denote the linear expression
 $\vec{c}^T\vec{y}+\vec{d}^T\vec{z}+e$
 and let $p(\vec{x}')$ denote
 $\vec{c}^T\vec{y}'+\vec{d}^T\vec{z}'+e$
 \\
 if $\lambda > 0$: set $\phi := 
  \phi \wedge [(p(\vec{x}') \leq p(\vec{x}) < 0)\vee
  (p(\vec{x}') \geq p(\vec{x}) > 0)\vee
  (p(\vec{x}') = p(\vec{x}) = 0)]
 $
 \\
 if $\lambda < 0$: set $\phi := 
  \phi \wedge 
 [(p(\vec{x}) \leq p(\vec{x}') < 0)\vee
  (p(\vec{x}) \geq p(\vec{x}') > 0)\vee
  (p(\vec{x}') = p(\vec{x}) = 0)]$
\item\label{loop1tail}
 if there are more than one eigenvectors corresponding to the eigenvalue $\lambda$, then update $\phi$ or $E$ by generalizing the above
\item
 repeat Steps~(\ref{loop1head})--(\ref{loop1tail}) 
 for each pair $(\vec{c},\lambda)$
 of left eigenvalue and eigenvector of $A_1$ 
\item\label{loop2head}
 let $\vec{c}+\imath\vec{d}$ be a complex left eigenvector
 of $A_1$ corresponding to complex eigenvalue $\alpha+\imath\beta$
\item
 using simple algebraic manipulation (linear equation solving)
 as above, find 
 $\vec{c_1}$, $\vec{d_1}$, $e_1$ and $e_2$ s.t.
 if $p_1$ denotes $\vec{c}^T\vec{y}+\vec{c_1}^T\vec{z}+e_1$
 and
 if $p_2$ denotes $\vec{d}^T\vec{y}+\vec{c_2}^T\vec{z}+e_2$
 then
 \begin{eqnarray*}
 \frac{d}{dt}(p_1)
  & = &
  \alpha p_1 - \beta p_2
\\
 \frac{d}{dt}(p_2)
  & = &
  \beta p_1 + \alpha p_2
  \end{eqnarray*}
 let $p_1'$ and $p_2'$ denote the primed versions of $p_1,p_2$
\item\label{loop2tail}
 if $\alpha \leq 0$: set $\phi := \phi\wedge( p_1^2+p_2^2 \geq
 {p_1'}^2 + {p_2'}^2)$
 \\
 if $\alpha \geq 0$: set $\phi := \phi\wedge( p_1^2+p_2^2 \leq
 {p_1'}^2 + {p_2'}^2)$
\item
 repeat Steps~(\ref{loop2head})--(\ref{loop2tail}) for every
 complex eigenvalue eigenvector pair
\item
 set $\phi := \phi\wedge
 \bigwedge_{e_1,e_2\in E} e_1 = e_2$
\item return $\phi$
\end{enumerate}
\caption{Algorithm implemented in HybridSal relational abstracter
for computing relational abstractions of linear ordinary differential
equations.}\label{fig:algo}
\end{figure}

The following proposition states the correctness of the algorithm.
\begin{proposition}\label{prop:corr}
Given $(A,b)$, let $\phi$ be the output of 
procedure {\linODEabs} in
Figure~\ref{fig:algo}.
If $\ra_{cont}$ is the binary relation defining the
semantics of $\frac{d\vec{x}}{dt} = A\vec{x}+b$ 
with mode invariant $\mathit{True}$
(as defined in Equation~\ref{eqn:sem}), then
$\ra_{cont}  \;\subseteq\; \phi$.
\end{proposition}
\begin{proof}(Sketch)
First, let $p(\vec{x})$ be the linear expression
$\vec{c}^T\vec{y}+\vec{d}^T\vec{z}+e$ discovered in Step~(6).
Then,
\begin{eqnarray*}
\frac{dp}{dt} &=& \vec{c}^T(A_1\vec{y}+A_2\vec{z}+\vec{b_1}) +
 \vec{d}^T\vec{b_2}
\; = \; \lambda\vec{c}^T\vec{y}+\lambda\vec{d}^T\vec{z}+\vec{c}^T\vec{b_1} + \vec{d}^T\vec{b_2}
\\
& = & \lambda*(\vec{c}^T\vec{y}+\vec{d}^T\vec{z}+\vec{c})
\; = \; \lambda*p
\end{eqnarray*}
Hence, $p(\vec{x}(t)) = p(\vec{x}(0)) e^{\lambda t}$.
Therefore, the relation added in Step~(6) to $\phi$ 
will hold between an initial state $\vec{x}$ and a future
state $\vec{x'}$.

Next, consider the quadratic relations added to $\phi$ 
in Step~(11).  Let $p_1,p_2$ be as defined in Step~(10).
Then,
\begin{eqnarray*}
\frac{d(p_1^2+p_2^2)}{dt} 
& = &
 2p_1(\alpha p_1-\beta p_2)+2p_2(\beta p_1+\alpha p_2)
\; = \;
 2\alpha(p_1^2 + p_2^2)
\end{eqnarray*}
Hence, 
$p_1(\vec{x}(t))^2+p_2(\vec{x}(t))^2 = 
 (p_1(\vec{x}(0))^2+p_2(\vec{x}(0))^2) e^{2\alpha t}$,
and therefore, the relation added in Step~(11) to $\phi$
will hold between an initial state $\vec{x}$ and a future
state $\vec{x'}$.

Finally, consider the relations added in Step~(13).
It is easy to observe that every expression $s(\vec{x},\vec{x'})$ 
in the set $E$ is equal to the time $t$ taken to reach $\vec{x}'$
from $\vec{x}$ following the linear ODE dynamics.
Hence, all these expressions need to be equal, as stated in Step~(13).
\qed
\end{proof}

By applying the above abstraction procedure on to the
dynamics of each mode of a given hybrid system,
the HybridSal relational abstracter constructs  a
relational abstraction of a hybrid system.
This abstract system is a purely discrete 
infinite state space system that can be analyzed
using infinite bounded model checking and 
k-induction.

We make two important remarks here.
First,
the relational abstraction constructed by 
procedure {\linODEabs} is a Boolean combination
of linear {\em{and nonlinear}} expressions.
The nonlinear expressions can be replaced by their
conservative linear approximations.  The
HybridSal relational abstracter performs this approximation
by default.  However, it can also generate the (more precise)
nonlinear abstraction (as described in Figure~\ref{fig:algo})
when invoked using an appropriate command line flag.
Both infinite bounded model checkers and k-induction
provers rely on satisfiability modulo theory (SMT) solvers.
Most SMT solvers can only reason with Boolean combination
of {\em{linear}} constraints, and hence, the ability to
generate linear relational abstractions is important.
However, there is significant research effort going on
into extending SMT solvers to handle nonlinear expressions.
HybridSal relational abstracter and SAL infinite bounded
model checker have been used to create benchmarks for 
linear {\em{and nonlinear}} SMT solvers.

Second, Procedure {\linODEabs} can be extended to
generate even more precise {\em{nonlinear}}
relational abstractions of linear systems.
Let $p_1, p_2, \ldots, p_k$ be $k$ (linear and nonlinear)
expressions found by Procedure {\linODEabs} that 
satisfy the equation $\frac{dp_i}{dt} = \lambda_i p_i$.
Suppose further that 
there is some $\lambda_0$ s.t.
for each $i$
$\lambda_i = n_i \lambda_0$ for some {\em{integer}} $n_i$.
Then, we can extend $\phi$ by adding the following relation to it:
$$
 p_i(\vec{x}')^{n_j} p_j(\vec{x})^{n_i}
 = p_j(\vec{x}')^{n_i}  p_i(\vec{x})^{n_j}
$$
The above relationship holds for any binary reachable
pair of states $(\vec{x},\vec{x'})$ 
because
$$
 \left(\frac{p_i(\vec{x}')}{p_i(\vec{x})}\right)^{n_j} 
 = 
 \left(\frac{p_j(\vec{x}')}{p_j(\vec{x})}\right)^{n_i} 
 =
 e^{n_in_j\lambda_0 t}
$$
However, since $p_i$'s are linear or quadratic 
expressions, the above relations will be highly
nonlinear unless $n_i$'s are small.  So, they are
not currently generated by the relational abstracter.
It is left for future work to see if good and useful
linear approximations of these highly nonlinear relations
can be obtained.

\section{The HybridSal Relational Abstracter}

The HybridSal relational abstracter tool is available
for download from {\url{http://www.csl.sri.com/~tiwari/}}.
The tool directory contains the sources as well as
documentation  and examples.

The input to the tool is a file containing a 
specification of a hybrid system and safety
properties.  The HybridSal language naturally
extends the SAL language by providing syntax for
specifying ordinary differential equations.
SAL is a guarded command language for specifying
discrete state transition systems and supports
modular specifications using synchronous and
asynchronous composition operators.  The reader
is referred to~\cite{SAL-language} for details.
HybridSal inherits all the language features of
SAL.  Additionally, HybridSal allows 
differential equations to appear in the model as follows:
for each real-valued variable $x$, the user 
defines a dummy variable $xdot$
which represents $\frac{dx}{dt}$.  A differential
equation can now be written by assigning to
the $xdot$ variable.
Assuming two variables $x,y$, the syntax is as follows:
\begin{quote}
\begin{tt}
 guard(x,y) AND guard2(x,x',y,y') --> xdot' = e1; ydot' = e2
\end{tt}
\end{quote}
This represents the system of differential equations
$\frac{dx}{dt} = e1, \frac{dy}{dt} = e2$ with
mode invariant $\mathit{guard}(x,y)$.
The semantics of this guarded transition is the binary
relation defined in Equation~\ref{eqn:sem} conjuncted with
the binary relation $\mathit{guard2}(x,x',y,y')$.
The semantics of all other constructs in HybridSal match
exactly the semantics of their counterparts in SAL.

Figure~\ref{fig:ex} contains sketches of a few examples of 
hybrid systems modeled in HybridSal.
The example in Figure~\ref{fig:ex}(left) defines a SAL module
{\tt{SimpleHS}} with two real-valued variables $x,y$.  Its 
dynamics are defined by differential equations 
$\frac{dx}{dt}=-y+x$, $\frac{dy}{dt}=-y-x$ with mode
invariant $y \geq 0$, and a 
discrete transition that resets the value of $x$ and $y$ when
$y \leq 0$.
The HybridSal file {\tt{SimpleHS.hsal}} also defines 
two safety properties. The latter
one says that $x$ is always non-negative.
This model is analyzed by first abstracting it 
\begin{quote}
 {\tt{ bin/hsal2hasal examples/SimpleEx.hsal}}
\end{quote}
to create a SAL file named {\tt{examples/SimpleEx.sal}},
and then (bounded) model checking the SAL file
\begin{quote}
 {\tt{ sal-inf-bmc -i -d 1 SimpleEx helper}}
\\
 {\tt{ sal-inf-bmc -i -d 1 -l helper SimpleEx correct}}
\end{quote}
The above commands prove the safety property using $k$-induction:
first we prove a lemma, named {\tt{helper}}, using 1-induction and then
use the lemma to prove the main theorem named {\tt{correct}}.

The example in Figure~\ref{fig:ex}(right) shows the sketch of
a model of the train-gate-controller example in HybridSal.
All continuous dynamics are moved into one module (named
{\tt{timeElapse}}).  The 
{\tt{train}},
{\tt{gate}} and
{\tt{controller}} modules define the state machines and are
pure SAL modules.
The {\tt{observer}} module is also a pure SAL module
and its job is to enforce synchronization between
modules on events.
The final system is a complex composition of the base modules.

\begin{figure}[htb!]
\begin{tt}
\begin{tabular}{lll}
\begin{minipage}{2in}
\begin{tabbing}
SimpleEx: CONTEXT = 
\\
BEGIN
\\
SimpleHS: MODULE =
\\
BEGIN
\\
LOCAL x,y:REAL
\\
LOCAL xdot,ydot:REAL
\\
IN\=ITIALIZATION
\\ \>
  x = 1; y IN \{z:REAL| z <= 2\}
\\
TRANSITION
\\
$[$
\\
y \= >= 0 AND y' >= 0 --> 
\\ \>
  xdot' = -y + x ;
\\ \>
  ydot' = -y - x
\\
$[]$
\\
y <= 0 --> x' = 1; y' = 2
$]$
\\
END;
\\
helper: LEMMA
\\
SimpleHS |- G(0.9239 * x >= 0.3827 * y);
\\
% proved using sal-inf-bmc -i -d 2 -l helper Linear1 correct
correct : THEOREM
\\
SimpleHS |- G(x >= 0);
\\
END
\end{tabbing}
\end{minipage}
&
\begin{minipage}{2in}
% Train gate controller
\begin{tabbing}
TGC: CONTEXT =
\\
BEGIN
\\
Mode: TYPE = \{s1, s2, s3, s4\};
\\
timeElapse: MODULE = 
\\
BEGIN
\\
 GLOBAL/LOCAL/INPUT {\em{variable declarations}}
% \\
 % LOCAL xdot, ydot, zdot: REAL
% \\
 % INPUT s, g, c: Mode
% \\
 % INPUT lower, exit, raise, approach: BOOLEAN
\\
 INITIALIZATION x = 0; y = 0; z = 0 
\\
 TRANSITION
\\
$[$
 {\em{mode invariants}}
 % ((s = s1 AND exit = FALSE) OR
% \\
  % (s = s2 AND approach = FALSE AND x <= 5 AND x' <= 5)) AND
% \\
 % ((g = s2 AND y <= 1 AND y' <= 1) OR
% \\
  % (g = s3 AND y <= 2 AND y' <= 2) OR (g = s1)) AND
% \\
 % ((c = s2 AND z <= 1 AND z' <= 1) OR
% \\
  % (c = s3 AND lower = FALSE) OR 
% \\
  % (c = s4 AND z <= 1 AND z' <= 1) OR
% \\
  % (c = s1 AND raise = FALSE)) AND
% \\
 % (x' >= x + 1/2) AND
% \\
 % (x >= 0 AND y >= 0 AND z >= 0 AND x' >= 0 AND y' >= 0 AND z' >= 0)
\\
 --> xdot' = 1; ydot' = 1; zdot' = 1 $]$
\\
END;
\\

\\
train: MODULE = $\ldots$
%\\
%BEGIN
%\\
  %GLOBAL/OUTPUT declarations
%% approach, exit: BOOLEAN
%%\\
  %%GLOBAL x: REAL
%%\\
  %%OUTPUT s: Mode
%\\
  %INITIALIZATION  s = s1; $\ldots$ %approach = FALSE; exit = FALSE
%\\
  %TRANSITION
%\\
  %$[$
%\\
  %s = s1 --> s' = s2; $\ldots$ % approach' = TRUE; x' = 0
%\\
  %$[]$
%\\
  %%s = s2 AND approach = FALSE AND x > 2 --> $\ldots$ %exit' = TRUE; s' = s1
%\\
  %$]$
%\\
%END ;
%\\

\\
gate: MODULE = $\ldots$
% \\
% BEGIN
% \\
  % GLOBAL/OUTPUT declarations
  % % GLOBAL lower, raise: BOOLEAN
% % \\
  % % GLOBAL y: REAL
% % \\
  % % OUTPUT g: Mode
% \\
  % INITIALIZATION  g = s1
% \\
  % TRANSITION
% \\
  % $[$
  % g = s1 AND lower = TRUE 
% \\
 % --> lower' = FALSE; g' = s2; y' = 0
% \\
  % $[] \ldots [] \ldots ]$
% %\\
  % %g = s2 AND raise = TRUE --> g' = s3; raise' = FALSE; y' = 0
% %\\
  % %[]
% %\\
  % %g = s3 AND y >= 1 --> g' = s1
% %\\
  % %]
% \\
% END ;
%\\

\\
controller: MODULE = $\ldots$
% \\
% BEGIN
% \\
  % GLOBAL/OUTPUT declarations
% % lower, raise, approach, exit: BOOLEAN
% %\\
  % %GLOBAL z: REAL
% %\\
  % %OUTPUT c: Mode
% \\
  % INITIALIZATION  c = s1; $\ldots$ % lower = FALSE; raise = FALSE
% \\
  % TRANSITION
% \\
  % $[$
% \\
  % c = s1 AND raise = FALSE AND approach = TRUE 
% \\
  % --> approach' = FALSE; c' = s2; z' = 0
% % \\
  % % $[]$
% % \\
  % % c = s2 AND z = 1 --> c' = s3; lower' = TRUE
% \\
  % $[] \ldots [] \ldots [] \ldots ]$
% %\\
  % %c = s3 AND lower = FALSE AND exit = TRUE --> c' = s4; z' = 0; exit' = FALSE
% %\\
  % %[]
% %\\
  % %c = s4 --> c' = s1; raise' = TRUE
% %\\
  % %]
% \\
% END;
%\\

\\
observer: MODULE = $\ldots$
%\\
%BEGIN
%\\
  %INPUT lower, raise, approach, exit: BOOLEAN
%\\
  %TRANSITION
%\\
  %$[$
%\\
  %NOT(lower AND lower') AND NOT(raise AND raise') AND 
%\\
  %NOT(approach AND approach') AND NOT(exit AND exit') -->
%\\
  %$]$
%\\
%END;
\\

\\
system: MODULE = (observer ||
\\
 (train [] gate [] controller [] timeElapse));
\\

\\
% sal-inf-bmc -i -ice -d 10  TGC correct
% The gate is down (g = s2) when the train is on the intersection (s = s2)
correct: THEOREM
\\
  system |- G( (s = s2 AND x >= 2) => g = s2 ) ;

% The gate is always closed: This is false
%\\
%canreach: THEOREM
%\\
  %system |- G( g = s2 ) ;

% The gate is always open: This is false
%\\
%canreach1: THEOREM
%\\
  %system |- G( g = s1 OR g = s3) ;
%
%
\\
END
\end{tabbing}
\end{minipage}
& 
\begin{minipage}{2in}
\end{minipage}
\end{tabular}
\end{tt}
\caption{Modeling hybrid systems in HybridSal: A few examples.}
\label{fig:ex}
\end{figure}

The above two examples, as well as,
several other simple examples are provided in the
HybridSal distribution to help
users understand the syntax and working of the
relational abstracter.  
A notable (nontrivial) example in the distribution is a hybrid
model of an automobile's automatic transmission from~\cite{Butts}.
Users have to separately
download and install SAL model checkers if they
wish to analyze the output SAL files using
k-induction provers or infinite bounded model checkers.

The HybridSal relational abstracter constructs abstractions
compositionally -- that is, it works on each mode 
(each system of differential equations) separately.
It just performs some simple linear algebraic manipulations
and is therefore very fast.  
The input HybridSal (.hsal) file is parsed into an XML (.hxml)
file, and the tool works by replacing the XML elements 
(in the XML DOM) representing
the continuous differential equations by new XML elements 
representing the relational abstraction. This creates a 
SAL XML file (.xml), which is pretty printed into a SAL (.sal) file. 
The bottleneck step in the analysis
is the (bounded) model checking or k-induction proving step
(which is orders of magnitude slower than the abstraction step).
The performance of HybridSal matches the performance reported 
in our earlier paper~\cite{ST:CAV2010} on the navigation
benchmarks (which are included with the HybridSal distribution).
In~\cite{ST:CAV2010} we had used many
different techniques (not all completely automated at that time)
to construct the relational abstraction.



%\section{Case Studies and Examples}
%
%illustrate 

\section{Discussion: Strengths and Weaknesses}

The HybridSal relational abstracter is a tool for verifying
hybrid systems.  The other common tools for hybrid
system verification consist of 
(a) tools that iteratively compute
an overapproximation of the reachable states~\cite{SpaceEx},
(b) tools that directly search for correctness certificates
(such as inductive invariants or Lyapunov function)~\cite{SOSTools,ISSAC11},or
(c) tools that compute an abstraction and then analyze
the abstraction~\cite{ }.
Our relational abstraction tool falls in category~(c), but unlike all
other abstraction tools, it does
not abstract the state space, but abstracts only the transition relation.
 
The key benefit of developing the concept of relational abstractions
is that it provides a new way to combine reasoning on continuous
dynamics (as developed in control theory or systems theory)
and
reasoning on discrete state transition systems (as developed in
the formal methods and computer science community.)
The former is used for constructing high quality relational abstractions
and the latter is used for verifying the abstract system.
Concepts such as Lyapunov functions or inductive invariants
(aka barrier certificates) for continuous systems can be used
to construct very precise relational abstractions of continuous
dynamical systems, which then helps in the final verification effort
of the abstracted system.
In fact, for several classes of simple continuous dynamical systems, 
lossless relational abstractions can be constructed, and hence
all incompleteness in verification then comes from incompleteness
of k-induction provers.

When the relational abstraction is analyzed using 
the k-induction prover or the infinite bounded model checker,
we generate large SMT formulas containing linear and nonlinear
arithmetic with Booleans connectives and other finite datatypes.
These instances can serve as benchmarks for SMT provers.
The scalability of verification is now strictly tied to the
scalability of SMT solving.

The relational abstraction methodology and tool have certain
weaknesses, which we now enumerate.  
(a) Relational abstraction generates verification 
problem on a discrete, infinite state space system, which are difficult to
automatically handle.  The successful application of k-induction 
can require the need for auxiliary lemmas.  The SMT formulas
generated can be very large.
(b) Our tool does not use the mode invariants when creating
relational abstractions.  The use of mode invariants can lead to
discovery of more relational invariants.  
(c) Our tool performs calculations using floating point 
arithmetic and hence the computed eigenvalues and eigenvectors
can have numerical errors.  
(d) There are other techniques for discovering relational
invariants that are not automated in our tool presently.
(e) Our tool is restricted to handling linear differential equations.
It can not handle nonlinear dynamics presently.
(f) Our tool can not efficiently handle platform constraints
imposed on control systems, such as sampling frequency, 
sensing and actuating delays, etc.  The theory of 
timed relational abstraction~\cite{ } is being developed for that
purpose.
We note that this is the first version of the tool and we hope to
enhance the tool to address some of the above concerns in 
future releases of the tool.



%\section{Introduction}
%\input{intro}
%\section{Preliminaries}\label{Section:Prelims}
%\input{prelims}
%\section{Relational Abstractions}
%\input{relationalAbstractions}
%\section{Implementation}\label{Section:Implementation}
%\input{implementation}
%\section{Experimental Evaluation}
%\input{experiments}

%\section{Conclusions}
%We have presented an approach for verifying hybrid systems based on
%relational abstractions.  Relational abstractions can be constructed
%compositionally by abstracting each mode separately.  Our initial
%results are quite encouraging. The technique successfully solves some
%of the standard benchmark examples an order of magnitude faster than
%symbolic model checkers. The abstractions can be coarse, and
%$k$-induction itself can be challenging to apply on hybrid systems in
%practice. Our future work will focus on improving the speed and
%precision of relational abstraction generation to enable fast proofs
%for complex systems. We also wish to apply our techniques to nonlinear
%hybrid systems in order to derive linear arithmetic abstractions.



\bibliographystyle{abbrv}
\bibliography{database}

%%\newpage %% For the on-line verification.
%%\appendix{Proofs of Claims}
%%\input{proofs}

\end{document}

